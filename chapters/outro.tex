\startconclusionpage

В данной работе была детально изучена структура данных Rake-Compress дерево. 
Она позволяет хранить информацию о лесе корневых деревьев, а также эффективно обрабатывать запросы его изменения (добавление и удаление ребер).

Реализация Rake-Compress деревьев, которая была описана в исходной статье, была существенно модифицирована. 
В частности, было показано как сократить расход памяти с $O(n \log n)$ до $O(n)$ без потери скорости работы.

Описанная в статье структура данных работала только в том случае, если степень каждой вершины ограничена некоторой константой. 
Было показано как избавиться от этого ограничения не ухудшив время работы.

Также было показано, как применить предложенные оптимизации в случае неориентированных деревьев. 

Кроме того, был предложен способ пересчета функций на лесе деревьев. 
Для аддитивных функций пересчет осуществляется за время $O(\log n)$, а для ассоциативных за $O(\log^2 n)$.

Структура данных Rake-Compress дерево с предложенными оптимизациями была реализована на языке программирования Java. 
Было проведено сравнение данного алгоритма, а также Link-Cut деревьев.

В дальнейшем планируется разработка алгоритма, который бы строил, а также перестраивал Rake-Compress деревья, используя несколько ядер одновременно.

\FloatBarrier
