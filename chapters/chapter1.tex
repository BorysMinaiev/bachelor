%-*-coding: utf-8-*-
\chapter{Обзор динамических деревьев}
\label{chapSVD}

В данной главе рассмотрены необходимые определения теории графов, 
поставлена задача, которая решается с помощью динамических деревьев, 
описаны области, в которых они применимы, а также 
приведен краткий обзор существующих динамических деревьев. 
В главе также выделены проблемы Rake-Compress деревьев, решению которых посвящена данная работа.

\section{Необходимые элементы теории графов}

Для начала введем несколько определений из теории графов, которые понадобятся в работе.

{\bf Графом} называется упорядоченная пара $(V, E)$, где $V$ --- множество вершин, а $E$ --- множество ребер.

Различают {\bf ориентированные} и {\bf неориентированные} графы. В первом случае каждое ребро это упорядоченная пара вершин, во втором --- неупорядоченная.

Вершина $v$ является {\bf смежной} с $u$, если в графе существует ребро $(u, v)$.

Говорят, что из $u$ {\bf достижима} $v$, если существует список вершин такой, что:
\begin{itemize}
\item первая вершина в списке --- $u$;
\item последняя вершина в списке --- $v$;
\item соседние вершины списка являются смежными.
\end{itemize}  

Граф называется {\bf связным}, если для любой пары вершин $(u, v)$ $u$ достижима из $v$.

{\bf Деревом} называется связный неориентированный граф с $n$ вершинами и $n - 1$ ребром.

{\bf Степенью вершины} называется количество вершин, смежных с данной. В случае ориентированных графов различают {\bf входящую} и {\bf исходящую} степени. 

{\bf Корневым деревом} называется ориентированный граф с выделенной вершиной --- корнем. Исходящая степень корня равна нулю, а исходящая степень любой другой вершины --- единице. Корень должен быть достижим из всех вершин.

{\bf Поддеревом} вершины $v$ называется множество всех вершин, из которых достижима $v$.

{\bf Листом} называется вершина, входящая степень которой равна нулю.

{\bf Лесом} называется множество деревьев, которые не пересекаются по вершинам.

В данной работе большинство внимания уделено эффективному хранению леса, каждое дерево которого является корневым. 
Однако в разделе \ref{sec:undirected} рассмотрен случай неориентированных деревьев.

\FloatBarrier

\section{Динамические деревья}

Задача, которая решается с помощью динамических деревьев, формулируется следующим образом. Необходимо поддерживать лес деревьев и выполнять на нем следующие операции:
\begin{itemize}
\item Добавить ребро $(u, v)$. Вершина $u$ должна быть корнем некоторого дерева. Вершины $u$ и $v$ должны находиться в разных деревьях.
\item Удалить ребро $(u, v)$. Ребро $(u, v)$ должно присутствовать в графе.
\item Некоторый запрос относительно структуры дерева.
\end{itemize}

Примером последней операции может быть запрос ``достижима ли вершина $u$ из $v$?'', 
``сколько ребер на кратчайшем пути из $u$ в $v$?'' или ``какова сумма номеров вершин, которые находятся в поддереве вершины $u$?''.
Можно легко реализовать структуру данных, которая будет выполнять данные
 операции за время $O(n)$, где $n$ --- количество вершин в графе. 
Динамические деревья нужны для того, чтобы обрабатывать запросы более эффективно. 
В частности, все предложенные операции можно выполнять за время $O(\log n)$.
  
\FloatBarrier

\section{Применение динамических деревьев}

Динамические деревья находят свое применение во многих областях информатики. Рассмотрим некоторые примеры:
\begin{itemize}
\item Используются динамических деревьев в алгоритме проталкивания предпотока, что уменьшает время его работы с 
$O(V^2 E)$ до $O(VE \log (V^2 / E))$, а время работы алгоритма Диница уменьшается с $O(V^2 E)$ до $O(VE \log V)$~\cite{tarj83}.
\item Используются для динамической локализации точек на плоскости~\cite{loca91}.
\item Используются для динамизации статических алгоритмов~\cite{acar05}. Рассмотрим пример такого использования. 
Существуют алгоритмы для подсчета значений арифметических выражений, которые основываются на построении двоичных деревьев разбора. 
Используя динамические деревья можно эффективно пересчитывать значения выражений при их незначительном изменении
(замены чисел в выражении или изменении его структуры).
\item Используются для динамического пересчета минимальных остовных деревьев~\cite{fred85}.
\end{itemize}

На практике использование динамических деревьев часто оказывается неэффективным из-за большой константы,
которая скрыта в асимптотике. Поэтому задача уменьшения константы в алгоритмах динамических деревьев является актуальной.
                                                                                                                                                                 
\FloatBarrier

\section{Существующие динамические деревья}

Самыми известными динамическими деревьями являются Link-Cut деревья, которые были предложены Слетером и Тарьяном в 1983 году. 
В них каждое дерево представляется набором путей, которые не пересекаются по вершинам. Каждый такой путь в свою очередь хранится в сбалансированном дереве поиска.
Если в качестве такого дерева использовать Splay-дерево, то амортизированно каждая операция с Link-Cut деревом будет выполнена за $O(\log n)$.
Такие деревья позволяют пересчитывать некоторые функции на путях, но не позволяют считать функции на поддереве.

Другим примером динамических деревьев является Top-деревья. Основная их идея заключается в построении сбалансированного двоичного дерева, в котором каждому листу соответствует ребро исходного дерева,
а внутренней вершине --- подмножество вершин исходного дерева.
С помощью этого дерева можно находить наибольшее ребро на пути между вершинами за асимптотическую сложность $O(\log n)$, а также наибольшее ребро в поддереве вершины за такую же сложность.
Кроме того с помощью них можно пересчитывать диаметр и центр дерева, а также некоторую другую информацию о структуре дерева.

Еще один тип динамических деревьев --- Rake-Compress деревья. Их отличительной особенностью является возможность их параллельного построения. 
Однако реализация, которая была описана в оригинальной статье, обладала существенными недостатками. 
Например, их можно было строить только на деревьях, у которых степень каждой вершины ограничена константой. 
А константа, которая скрыта в асимптотике времени работы и используемой памяти, сильно мешает их использованию на практике.

Подробное сравнение динамических деревьев проведено в~\cite{tarj07}.

\FloatBarrier

\section{Постановка задачи}

Rake-Compress дерево является перспективной структурой данных с точки зрения возможности их распараллеливания. 
Однако, на текущий момент они находят мало применений в реальной жизни из-за сильно большой константы времени работы и используемой памяти.
Из-за невозможности их использования на произвольных деревьях, на практике их место занимают другие динамические деревья.
Целью данной работы является избавление Rake-Compress деревьев от перечисленных недостатков. 



\FloatBarrier
%%% Local Variables:
%%% mode: latex
%%% TeX-master: t
%%% End:
