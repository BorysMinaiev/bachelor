% -*-coding: utf-8-*-
\startprefacepage

Динамические деревья находят множество применений в области информационных технологий. В частности, они используются в 
алгоритмах поиска максимального потока, а также для решения задачи динамической 
связности ациклических графов. Например, они позволяют уменьшить время работы алгоритма проталкивания
предпотока с $O(EV^2)$ до $O(EV \log (V^2 / E))$~\cite{tarj86}. 

Зачастую, термин <<динамические деревья>> ассоциируется только с Link-Cut деревьями, которые были предложены 
Слетером и Тарьяном~\cite{tarj83}. Однако существуют и другие динамические деревья: например, Top-деревья~\cite{alst05} и Rake-Compress деревья~\cite{acar04, acar05}.
В данной работе рассмотрены именно последние. Они обладают существенным преимуществом перед другими динамическими 
деревьями --- их построение можно сделать параллельным.

На сегодняшний день большинство процессоров содержат несколько ядер, способных работать параллельно. 
Поэтому актуальной задачей является разработка алгоритмов, способных работать на нескольких ядрах одновременно.
Алгоритмы, базирующиеся на Rake-Compress деревьях, являются таковыми.

Реализация Rake-Compress деревьев, которая была предложена Умутом Акаром в оригинальной статье~\cite{acar04} обладала несколькими недостатками:
\begin{itemize}
\item Рассматривались только деревья, у которых степень каждой вершины ограничена некоторой константой. 
\item Статья носит скорее теоретический, нежели практический характер. В ней уделяется внимание только асимптотическим оценкам. 
На практике же необходимо иметь структуры данных, эффективные не только асимптотически, но и с точки зрения времени работы на реальных данных. 
\end{itemize}

В данной работе показано как избавиться от ограничения на степени вершин, а также как эффективно реализовывать Rake-Compress деревья.

\FloatBarrier
