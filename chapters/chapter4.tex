\chapter{Детали реализации и сравнение с аналогами}

В данной главе подробно рассмотрены детали реализации динамических Rake-Compress деревьев, а также приведен сравнительный анализ деревьев Rake-Compress и Link-Cut.

\FloatBarrier
\section{Генерация псевдослучайных бит}    
  
Для реализации Rake-Compress деревьев необходимо уметь генерировать случайные биты для определения, к каким вершинам можно применить операцию Compress.
Более формально: необходима структура данных, которая на вход принимает общее количество вершин и генератор псевдослучайных чисел, 
а на выход для каждой пары ($v$, $layer$) выдает случайный бит. 
При этом, для одинаковых входных данных, биты, которые генерируются, должны быть одинаковы вне зависимости от того, в каком порядке их спрашивают.
Также желательно чтобы данная структура данных занимала $O(n)$ памяти.

Воспользуемся оптимизацией, о которой говорится в разделе~\ref{sec:memory}. 
А именно, создадим генераторы псевдослучайных чисел для каждой вершины отдельно и будем отвечать на запросы лениво, сохраняя ответы в саморасширяющемся массиве (отдельном для каждой вершины).


\FloatBarrier
\begin{algorithm}
\caption{Структура данных для генерации случайных бит}\label{algo:random_gen}
\begin{algorithmic}[1]
\State $RandomBitsGenerator(int \ n, Random \ rnd)$
\For{$i = 0$ to $n$}
	\State $rand[i] \gets new \ Random(rnd.nextInt())$ 
	\State $bits[i] \gets \emptyset$
\EndFor

\Procedure{getBit}{$v, layer$}
	\While {$bits[v].size() \leq layer$}
		\State $bits[v].add(rand[v].nextBoolean())$
	\EndWhile
	\State \textbf{return} $bits[v].get(layer)$
\EndProcedure

\end{algorithmic}
\end{algorithm}

\FloatBarrier
\section{Хранение клеток таблицы Rake-Compress дерева}      

Рассмотрим более подробно, как необходимо хранить клетки таблицы Rake-Compress дерева. 
Для вершины необходимо сохранить ее родителя, а также множество детей. 
Как уже говорилось в разделе~\ref{sec:set_storage} множество детей хранится как сумма их номеров, а также их количество.
Если вершина является корнем, то в качестве ее родителя будем хранить ее номер.
Кроме того необходимо хранить изменения, которые произойдут с клеткой при переходе к следующему слою.

\FloatBarrier
\begin{algorithm}
\caption{Хранение клеток таблицы Rake-Compress дерева}\label{algo:cell}
\begin{algorithmic}[1]
\State \textbf{int} $id, parent, sumChild, cntChild$
\State \textbf{int} $newParent, diffSumChild, diffCntChild$

\Statex
\Procedure{applyChanges}{}
	\State $parent \gets newParent$
	\State $sumChild \gets sumChild + diffSumChild$
	\State $cntChild \gets cntChild + diffCntChild$
	\State $diffSumChild \gets 0$
	\State $diffCntChild \gets 0$
\EndProcedure

\Statex
\Procedure{addChildToDiff}{$v$}
	\State $diffSumChild \gets diffSumChild + v$
	\State $diffCntChild \gets diffCntChild + 1$
\EndProcedure

\Statex
\Procedure{removeChildFromDiff}{$v$}
	\State $diffSumChild \gets diffSumChild - v$
	\State $diffCntChild \gets diffCntChild - 1$
\EndProcedure

\end{algorithmic}
\end{algorithm}

\FloatBarrier
\section{Хранение Rake-Compress дерева}   

Для каждой вершины необходимо хранить список клеток, которые ей соответствуют, и 
 генератор псевдослучайных бит. 
Также заведем счетчик количества примененных операций по изменению структуры леса.
Еще будем хранить массив, в котором для каждой вершины запишем номер последней операции, 
при обработке которой была изменена хотя бы одна клетка, которая соответствуют вершине.
Это позволит эффективно узнавать, была ли вершина уже помечена как поменявшаяся или нет.

\FloatBarrier
\begin{algorithm}
\caption{Хранение Rake-Compress дерева}\label{algo:rc_tree_storage}
\begin{algorithmic}[1]
\State $RakeCompressTree(int \ n, Random \ rnd)$
\State $RandomBitsGenerator \ rand \gets new \ RandomBitsGenerator(n, rnd)$
\State $time \gets 0$
\State $lastUpdateTime \gets \{0, \ldots, 0\}$
\For{$i = 0$ to $n$}
	\State $cells[i] \gets new \ List<Cell>$ 
\EndFor

\end{algorithmic}
\end{algorithm}


\FloatBarrier
\section{Построение Rake-Compress дерева}   

Рассмотрим, как работает алгоритм построения Rake-Compress дерева. Будем строить таблицу по строкам.
В каждый момент будем хранить множество вершин, которые еще не были сжаты, и перестраивать следующий слой.
Также будем делать операции Rake и Compress одновременно. Чтобы определить, нужно ли сжимать вершину, 
воспользуемся следующим алгоритмом:

\FloatBarrier
\begin{algorithm}
\caption{Определение необходимости сжатия вершины}\label{algo:check_rake_compress}
\begin{algorithmic}[1]
\Procedure{shouldRemoveVertex}{$Cell \ c, RandomBitsGenerator \ rand, int \ layer$}
	\If {$c.cntChild = 0$}
		\State \textbf{return} \algorithmictrue \Comment{После применения операции Rake}
	\EndIf
	\If {$c.cntChild > 1$ \algorithmicor $c.parent = c.id$}
		\State \textbf{return} \algorithmicfalse
	\EndIf
	\If {$getCellForVertex(c.sumChild).cntChild = 0$}
		\State \textbf{return} \algorithmicfalse \Comment{Операция Rake была применена к ребенку}
	\EndIf
	\If {$rand.getBit(c.id, layer) = 0$ \algorithmicand $rand.getBit(c.sumChild, layer) = 1$ \algorithmicand $rand.getBit(c.parent, layer) = 1$}
		\State \textbf{return} \algorithmictrue
	\EndIf
	\State \textbf{return} \algorithmicfalse
\EndProcedure

\end{algorithmic}
\end{algorithm}

Общий алгоритм построения Rake-Compress дерева выглядит следующим образом:

\FloatBarrier
\begin{algorithm}
\caption{Алгоритм построения Rake-Compress дерева}\label{algo:building_rc}
\begin{algorithmic}[1]
\State $alive \gets \{0, \ldots, n - 1\}$
\State $layer \gets 0$
\For{$i = 0$ to $n$}
	\State $cells[i].add(new \ Cell(parent[i]))$ 
\EndFor
\While {$alive \neq \emptyset$}
	\State $nextAlive \gets \emptyset$
	\For {$v \in alive$} 
		\State $c \gets getCellForVertex(v)$
		\If {$shouldRemoveVertex(c, rand, layer)$}
			\If {$c.cntChild = 1$}
				\State $getCellForVertex(c.sumChild).newParent \gets c.parent$
				\State $getCellForVertex(c.parent).addChildToDiff(c.sumChild)$
			\EndIf
			\If {$c.parent \neq v$}
				\State $getCellForVertex(c.parent).removeChildFromDiff(v)$
			\EndIf
		\Else
			\State $nextAlive.add(v)$
		\EndIf
	\EndFor
	\State $alive \gets nextAlive$
	\For {$v \in alive$}
		\State $newCell \gets getCellForVertex(v).clone().applyChanges()$ 
		\State $cells[v].add(newCell)$
	\EndFor
	\State $layer \gets layer + 1$
\EndWhile
\end{algorithmic}
\end{algorithm}

\FloatBarrier
\section{Изменение Rake-Compress дерева при удалении или добавлении одного ребра}   

Рассмотрим, что происходит с таблицей Rake-Compress дерева при изменении одного ребра.
Основная идея заключается в том, чтобы научится пересчитывать все изменения таблицы за время пропорциональное их количеству.
Для этого будем эффективно поддерживать множество изменившихся клеток. 
В момент, когда вершина помечается как изменившаяся, найдем, как она влияет на таблицу и отменим это влияние. 
Для начала необходимо найти момент времени, когда вершина сжимается. В этот момент она влияет не более чем на две вершины. 
Изменим значения $cntChild$, $sumChild$ и $newParent$ нужным образом. 
Также необходимо добавить эти вершины в множество изменившихся (в момент, когда будет обработан соответствующий слой).
Поэтому для каждого слоя еще будем хранить список вершин, которые должны быть помечены перед обработкой слоя.

Алгоритм обновления дерева будет выглядеть следующим образом:


\FloatBarrier
\begin{algorithm}
\caption{Алгоритм перестроения Rake-Compress дерева при удалении или добавлении ребра}\label{algo:rebuilding_rc}
\begin{algorithmic}[1]
\State $time \gets time + 1$
\State $affected \gets \emptyset$
\State $markAffected(v)$ \Comment {Пусть ребро $(u, v)$ было удалено}
\State $markAffected(u)$
\State $cells[u].parent \gets u$
\State $cells[v].cntChild \gets cells[v].cntChild - 1$
\State $cells[v].sumChild \gets cells[v].sumChild - u$
\State $layer \gets 0$
\While {$affected \neq \emptyset$}
	\For {$v \in affectedOnLayer[layer]$}
		\State $markAffected(v)$
	\EndFor
	\For {$v \in affected$} 
		\State $c \gets getCellForVertex(v)$
		\If {$shouldRemoveVertex(c, rand, layer)$}
			\State $cells[v].size \gets layer + 1$
			\If {$c.cntChild = 1$}
				\State $getCellForVertex(c.sumChild).newParent \gets c.parent$
				\State $getCellForVertex(c.parent).addChildToDiff(c.sumChild)$
				\State $markAffected(c.sumChild)$
			\EndIf
			\If {$c.parent \neq v$}
				\State $getCellForVertex(c.parent).removeChildFromDiff(v)$
				\State $markAffected(c.parent)$
			\EndIf
			\State $affected \gets affected \setminus v$
		\EndIf
	\EndFor
	\For {$v \in affected$}
		\State $newCell \gets getCellForVertex(v).clone().applyChanges()$ 
		\State $cells[v][layer + 1] \gets newCell$
	\EndFor
	\State $layer \gets layer + 1$
\EndWhile

\Statex

\Procedure{markAffected}{$int \ v$}
	\If{$lastUpdateTime[v] = time$}
		\State \textbf{return} \Comment {Вершина уже помечена}
	\EndIf
	\State $lastUpdateTime[v] \gets time$
	\State $affected \gets affected \cup v$
	\State $removeEffectOfVertex(v)$
\EndProcedure
\Procedure{removeEffectOfVertex}{$int \ v$}
	\State $layer \gets cells[v].size()$
	\State $c \gets cells[v].get(layer)$
	\If {$c.parent = v$}
		\State \textbf{return}
	\EndIf
	\State $cells[c.parent].removeChildFromDiff(v)$
	\If {$c.cntChild = 1$}
		\State $cells[c.parent].addChildToDiff(c.sumChild)$
		\State $cells[c.sumChild].newParent \gets v$
	\EndIf
\EndProcedure

\end{algorithmic}
\end{algorithm}

\FloatBarrier
\section{Сравнение с Link-Cut деревьями}   

Алгоритм был реализован на языке программирования Java, с его исходным кодом можно ознакомиться в \cite{github}.
Был проведен сравнительный анализ производительности разработанного алгоритма с алгоритмом Link-Cut дерева. 
В среднем на тестах, которые были проведены, Rake-Compress деревья оказываются медленнее в 5-10 раз чем Link-Cut.

Из этого можно сделать вывод, что в задачах, в которых применимы как Rake-Compress деревья, так и Link-Cut, следует использовать последние.
Однако, если в задаче требуется пересчитывать значения функций на поддеревьях, 
а не на путях, то Link-Cut деревья оказываются не применимы, и в таком случае можно воспользоваться Rake-Compress деревьями.
Кроме того, если удастся реализовать алгоритм, пересчитывающий Rake-Compress деревья и использующий несколько ядер одновременно, то, возможно, 
Rake-Compress деревья станут более востребованными.




